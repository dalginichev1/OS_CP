
\section{Архитектура проекта}

Архитектура проекта реализует клиент-серверную модель с использованием разделяемой памяти (shared memory) для межпроцессного взаимодействия. Система состоит из трёх ключевых компонентов, определённых в файлах проекта:

\begin{itemize}
    \item \textbf{Сервер} (\texttt{Server.hpp}, \texttt{Server.cpp}) — центральный координатор, который управляет всеми игровыми сессиями, обрабатывает команды клиентов и поддерживает целостность игрового состояния. Сервер работает в бесконечном цикле, ожидая сообщений в общей очереди.
    
    \item \textbf{Клиенты} (\texttt{Client.hpp}, \texttt{Client.cpp}) — независимые процессы, предоставляющие консольный интерфейс игрокам. Каждый клиент подключается к существующей разделяемой памяти, регистрируется с уникальным логином и взаимодействует с системой через меню.
    
    \item \textbf{Разделяемая память} (\texttt{SharedTypes.hpp}, \texttt{SharedMemory.hpp}, \texttt{SharedMemory.cpp}) — общая область памяти, содержащая структуру \texttt{SharedMemoryRoot} с очередью сообщений, слотами клиентов и массивами игр. Этот компонент обеспечивает высокоскоростное взаимодействие между процессами.
\end{itemize}

Все компоненты взаимодействуют через единую структуру данных, определённую в \texttt{SharedTypes.hpp}, которая включает типы сообщений (\texttt{MsgType}), состояния ячеек (\texttt{CellState}), кораблей (\texttt{Ship}) и игр (\texttt{GameData}). Синхронизация осуществляется с помощью POSIX мьютексов и условных переменных, также определённых в структуре разделяемой памяти.

\section{Метод решения}

Метод решения основан на использовании POSIX Shared Memory для реализации межпроцессного взаимодействия между сервером и клиентами. В файле \texttt{SharedTypes.hpp} определена центральная структура \texttt{SharedMemoryRoot}, содержащая:

\begin{itemize}
    \item \textbf{Очередь сообщений} (\texttt{Message queue[QUEUE\_SIZE]}) — циклический буфер для асинхронной передачи команд от клиентов серверу и обратно. Типы сообщений определены в перечислении \texttt{MsgType} и включают все возможные действия: регистрация (\texttt{MSG\_REGISTER}), создание игры (\texttt{MSG\_CREATE}), выстрел (\texttt{MSG\_SHOT}) и др.
    
    \item \textbf{Слоты клиентов} (\texttt{ClientSlot clients[MAX\_CLIENTS]}) — информация о каждом подключённом игроке, включая логин, условную переменную для ожидания ответов и текущий статус.
    
    \item \textbf{Игровые сессии} (\texttt{GameData games[16]}) — полное состояние каждой игры, включая поля игроков (\texttt{board1}, \texttt{board2}), массивы кораблей (\texttt{ships1}, \texttt{ships2}), статистику и временные метки.
\end{itemize}

\textbf{Серверный процесс} (реализованный в \texttt{Server.cpp}) работает по следующему алгоритму:

\begin{enumerate}
    \item Инициализирует разделяемую память и создаёт необходимые объекты синхронизации.
    \item Входит в основной цикл, ожидая сообщений через условную переменную \texttt{server\_cond}.
    \item При получении сообщения определяет его тип и вызывает соответствующий обработчик (\texttt{handle\_message()}).
    \item Обновляет состояние игр, отправляет ответы клиентам и синхронизирует изменения.
\end{enumerate}

\textbf{Клиентский процесс} (\texttt{Client.cpp}) реализует:

\begin{enumerate}
    \item Подключение к существующей разделяемой памяти.
    \item Регистрацию с уникальным логином через сообщение \texttt{MSG\_REGISTER}.
    \item Взаимодействие с пользователем через контекстные меню (главное меню, меню расстановки, игровое меню).
    \item Отправку команд в очередь и ожидание ответов через условные переменные.
\end{enumerate}

\textbf{Игровая логика} инкапсулирована в классе \texttt{Game} (\texttt{Game.hpp}, \texttt{Game.cpp}), который отвечает за:

\begin{itemize}
    \item Валидацию расстановки кораблей (проверка границ, пересечений, расстояний).
    \item Обработку выстрелов с определением попаданий, промахов и уничтожения кораблей.
    \item Управление очерёдностью ходов (правило дополнительного хода при попадании).
    \item Определение условия победы (уничтожение всех кораблей противника).
\end{itemize}

\textbf{Синхронизация} обеспечивается через мьютекс \texttt{mutex} в структуре \texttt{SharedMemoryRoot}. Каждая операция чтения или записи в разделяемую память предваряется захватом этого мьютекса, что предотвращает состояния гонки (race conditions). Условные переменные (\texttt{server\_cond} и \texttt{cond} в каждом \texttt{ClientSlot}) используются для эффективного ожидания событий без активного опроса.

\textbf{Обработка ошибок} реализована на всех уровнях: проверка корректности входных данных (координаты в пределах 0-9, правильный формат команд), валидация игровых действий (очередность хода, завершённость расстановки), обработка исключительных ситуаций (отключение клиента, переполнение очереди).

\section{Описание программы}

\subsection{Сервер (server)}

Сервер запускается первым и выполняет следующие функции:

\begin{itemize}
    \item Создаёт и инициализирует разделяемую память (если не существует).
    \item Устанавливает структуры данных: очередь сообщений, слоты клиентов, игровые сессии.
    \item Инициализирует мьютексы и условные переменные с атрибутами \texttt{PTHREAD\_PROCESS\_SHARED}.
    \item Ожидает команды от клиентов в бесконечном цикле, обрабатывая их в порядке поступления.
    \item Управляет жизненным циклом игр: создание, наполнение игроками, проведение игры, завершение и очистка.
    \item Обеспечивает целостность данных и соблюдение правил игры.
\end{itemize}

\subsection{Клиент (client)}

Клиентское приложение предоставляет пользователю консольный интерфейс с контекстными меню:

\begin{enumerate}
    \item \textbf{Главное меню} (отображается когда игрок не в игре):
    \begin{itemize}
        \item Список игроков и игр (команда \texttt{list} → \texttt{MSG\_LIST})
        \item Создание публичной игры (ввод имени → \texttt{MSG\_CREATE})
        \item Присоединение к игре по имени или ID (\texttt{MSG\_JOIN})
        \item Приглашение другого игрока по логину (\texttt{MSG\_INVITE})
        \item Проверка приглашений
        \item Выход из системы (\texttt{MSG\_QUIT})
    \end{itemize}
    
    \item \textbf{Меню расстановки кораблей} (после входа в игру):
    \begin{itemize}
        \item Ручное размещение: \texttt{place размер,x,y,ориентация} → \texttt{MSG\_PLACE\_SHIP}
        \item Автоматическая расстановка: \texttt{auto} (алгоритм в \texttt{Client.cpp})
        \item Просмотр своего поля: \texttt{board} → \texttt{MSG\_GET\_BOARD}
        \item Завершение расстановки: \texttt{ready} → \texttt{MSG\_SETUP\_COMPLETE}
        \item Приглашение в текущую игру: \texttt{invite логин} → \texttt{MSG\_INVITE\_TO\_GAME}
        \item Выход из игры: \texttt{menu} → \texttt{MSG\_LEAVE\_GAME}
    \end{itemize}
    
    \item \textbf{Игровое меню} (после начала боя):
    \begin{itemize}
        \item Сделать выстрел: ввод координат → \texttt{MSG\_SHOT}
        \item Просмотр своего поля → \texttt{MSG\_GET\_BOARD}
        \item Просмотр поля противника → \texttt{MSG\_GET\_OPPONENT\_BOARD}
        \item Статус игры → \texttt{MSG\_GAME\_STATUS}
        \item Сдаться → \texttt{MSG\_SURRENDER}
        \item Выйти в меню → \texttt{MSG\_LEAVE\_GAME}
    \end{itemize}
\end{enumerate}

\subsection{Игровой процесс}

Игра следует классическим правилам "Морского боя":

\begin{itemize}
    \item Поле 10×10 клеток.
    \item Флот из 10 кораблей: 1×4, 2×3, 3×2, 4×1 клетки.
    \item Расстановка с расстоянием минимум 1 клетка между кораблями.
    \item Поочерёдные ходы с правом дополнительного хода при попадании.
    \item Отображение полей: своё поле показывает корабли, поле противника скрывает неподбитые корабли.
    \item Победа при уничтожении всех кораблей противника.
\end{itemize}

\section{Особенности реализации}

\begin{itemize}
    \item \textbf{Автоматическая расстановка} использует алгоритм случайного размещения с проверкой корректности позиции.
    \item \textbf{Система приглашений} поддерживает два сценария: создание новой приватной игры и приглашение в существующую игру.
    \item \textbf{Визуализация} полей с помощью символов ASCII: \texttt{.} — пусто, \texttt{S} — корабль (на своём поле), \texttt{X} — попадание, \texttt{O} — промах, \texttt{\#} — потопленный корабль.
    \item \textbf{Обработка отключений}: сервер обнаруживает неактивных клиентов и корректно завершает игры.
    \item \textbf{Масштабирование}: ограничение на 32 одновременных клиента и 16 активных игр (константы в \texttt{SharedTypes.hpp}).
\end{itemize}