\section{Условие}
Консоль-серверная игра. Необходимо написать консоль-серверную игру. Необходимо написать 2 программы: сервер и клиент. Сначала запускается сервер, а далее клиенты соединяются с сервером. Сервер координирует клиентов между собой. При запуске клиента игрок может выбрать одно из следующих действий (возможно больше, если предусмотрено вариантом):
• 	Создать игру, введя ее имя
• 	Присоединиться к одной из существующих игр по имени игры


\subsection*{Цель работы}
1.	Приобретение практических навыков в использовании знаний, полученных в течении курса

2.	Проведение исследования в выбранной предметной области

\subsection*{Задание}
Необходимо спроектировать и реализовать программный прототип в соответствии с выбранным вариантом. Произвести анализ и сделать вывод на основании данных, полученных при работе программного прототипа.

\subsection*{Задание варианта}
Морской бой. Общение между сервером и клиентом необходимо организовать при помощи memory map. Каждый игрок должен при запуске ввести свой логин. Должна быть предоставлена возможность отправить приглашение на игру другому игроку по логину


\subsection*{Вариант} 6