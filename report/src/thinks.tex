\section{Выводы}

В процессе выполнения курсового проекта я получил практический опыт проектирования и реализации клиент-серверной системы с использованием разделяемой памяти для межпроцессного взаимодействия.

\textbf{Архитектура на основе shared memory} - позволяет достичь высокой производительности за счёт минимальных накладных расходов на передачу данных между процессами. Преимущества: быстрое взаимодействие, простота синхронизации через мьютексы, единое представление данных для всех компонентов. Недостатки: ограничение на количество процессов, сложность масштабирования на несколько машин, требования к корректной синхронизации.

\textbf{Клиент-серверная модель} - обеспечивает четкое разделение ответственности: сервер как координатор и валидатор, клиенты как пользовательские интерфейсы. Преимущества: централизованное управление состоянием, безопасность (проверка действий на сервере), возможность добавления новых клиентов без изменения сервера. Недостатки: единая точка отказа, зависимость от доступности сервера.

\subsection{Приобретенные навыки}

\begin{itemize}
    \item \textbf{Работа с POSIX shared memory} - освоил механизмы создания, отображения и управления разделяемой памятью в Linux
    \item \textbf{Межпроцессная синхронизация} - изучил использование мьютексов и условных переменных в многопроцессной среде
    \item \textbf{Проектирование протоколов взаимодействия} - разработал систему сообщений для клиент-серверного общения
    \item \textbf{Игровая логика и бизнес-правила} - реализовал полный цикл игры "Морской бой" с соблюдением всех правил
    \item \textbf{Консольный пользовательский интерфейс} - создал интуитивно понятные меню для взаимодействия с игроком
\end{itemize}

\subsection{Заключение}
Использование разделяемой памяти для реализации многопользовательской игры доказало свою эффективность для локальных систем. Такой подход особенно полезен в случаях, когда требуется:

\begin{itemize}
    \item Высокая производительность при интенсивном обмене данными
    \item Простота отладки и мониторинга (все данные в одном месте)
    \item Минимальные задержки между клиентами и сервером
\end{itemize}
Полученные знания и навыки позволяют создавать высокопроизводительные многопользовательские приложения с четкой архитектурой и надежной синхронизацией. Проект демонстрирует, что даже классические задачи (такие как "Морской бой") могут быть реализованы с использованием современных технологий межпроцессного взаимодействия, что открывает возможности для создания более сложных распределенных систем в будущем.
Особенно ценным оказался опыт работы с реальными проблемами синхронизации, обработки отказов и проектирования отказоустойчивых систем. Эти компетенции критически важны для разработки промышленного программного обеспечения.